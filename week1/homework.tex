
\documentclass[a4]{article}
\usepackage{graphicx}
\usepackage{amsmath}
\usepackage{quiver}
\usepackage{float}
\setlength{\parindent}{0pt}
\setlength{\parskip}{2pt}
\begin{document}

\section*{Aleksanteri Paakkinen 588807}

\textbf{Exercise 1} (1.3.D).
Verify that $A \rightarrow S^{-1}A$ satifies the following universal property:
$S^{-1}A$ is initial among $A$-algebras $B$ where every element of $S$ is mapped to an
invertible element of $B$.

Consider the following maps as presented in the Figure~\ref{localization}:
for $\phi: a \mapsto a/1$, $\alpha: A \rightarrow B$ we need to find $f: S^{-1}A
\rightarrow B$ s.t. $\alpha = f \circ \phi$

Elements of $S$ are mapped to invertible elements in both $S^{-1}A$ and $B$ so
we can freely take their inverses. % Let $s' = s^{-1}$ in $S^{-1}A$.

By properties of module homomorphisms we have $f(s/s) = sf(1/s) = 1 \implies
f(1/s) = f(s)^{-1}$

Using these facts we define $f$ to be
$f(a/s) = f(a/1)f(1/s) = \alpha(a)\left(f(1/s)\right) =
\alpha(a)\alpha(s)^{-1}$. Notice that $f$ is determined uniquely by $\alpha$ and
the diagram commutes: $(f \circ \phi) (a) =
f(a/1) = f(a)\cdot1 = \alpha(a)$.

\begin{figure}[H]
  % https://q.uiver.app/?q=WzAsMyxbMCwwLCJBIl0sWzEsMCwiU157LTF9QSJdLFsxLDEsIkIiXSxbMCwxLCJcXHBoaSJdLFswLDIsIlxcYWxwaGEiLDJdLFsxLDIsImYiLDAseyJzdHlsZSI6eyJib2R5Ijp7Im5hbWUiOiJkYXNoZWQifX19XV0=
\[\begin{tikzcd}
	A & {S^{-1}A} \\
	& B
	\arrow["\phi", from=1-1, to=1-2]
	\arrow["\alpha"', from=1-1, to=2-2]
	\arrow["f", dashed, from=1-2, to=2-2]
\end{tikzcd}\]
\caption{Universal property of localization}\label{localization}
\end{figure}

\vspace{0.5cm}
\textbf{Exercise 2} (1.3.H)
The classic thing happened and I ran out of time with this.

The exact sequence
\begin{align}
  M' \rightarrow M \rightarrow M'' \rightarrow 0
\end{align}
implies a surjective map $f: M \rightarrow M''$ and a map $f': M' \rightarrow
M$ which is surjective to kernel of $f$.

\vspace{0.5cm}
\textbf{Exercise 3} (1.3.Q)
The diagram commutes, thus the maps $U \rightarrow V \rightarrow X \rightarrow Z$ and $U \rightarrow W \rightarrow Y \rightarrow Z$ argee. Suppose we have an object E with maps to Y and U that agree in Z. By the upper cartesian square we have an unique map from E to U and conposing with $U \rightarrow W$ a unique map from $E$ to $W$. Additionally by composing maps we get a map $E \rightarrow X$.

E must factor uniquely through W, otherwise we would get a contradiction with the lower cartesian square. Furthermore by decomposing some of the compositions we made E factors uniquely through U implying that the outer square is a cartesian diagram.

\begin{figure}[H]
  % https://q.uiver.app/?q=WzAsMTUsWzExLDJdLFsxLDEsIlUiXSxbMywxLCJWIl0sWzEsMywiVyJdLFszLDMsIlgiXSxbMSw1LCJZIl0sWzMsNSwiWiJdLFswLDAsIkUiXSxbNSwzLCJXIl0sWzcsMywiWCJdLFs3LDUsIloiXSxbNSw1LCJZIl0sWzQsMSwiRSJdLFs2LDIsIlYiXSxbNSwyLCJVIl0sWzEsMl0sWzIsNF0sWzEsM10sWzMsNV0sWzUsNl0sWzQsNl0sWzMsNF0sWzcsNV0sWzcsMl0sWzcsMSwiIiwxLHsic3R5bGUiOnsiYm9keSI6eyJuYW1lIjoiZG90dGVkIn19fV0sWzEyLDExLCIiLDEseyJjdXJ2ZSI6Mn1dLFs4LDExXSxbMTEsMTBdLFs4LDldLFs5LDEwXSxbMTIsMTNdLFsxMyw5XSxbMTIsMTQsIiIsMSx7InN0eWxlIjp7ImJvZHkiOnsibmFtZSI6ImRvdHRlZCJ9fX1dLFsxNCw4XSxbMTIsOSwiIiwxLHsiY3VydmUiOi01fV0sWzEyLDgsIiIsMSx7InN0eWxlIjp7ImJvZHkiOnsibmFtZSI6ImRvdHRlZCJ9fX1dLFsxNCwxM11d
\[\begin{tikzcd}
	E \\
	& U && V & E \\
	&&&&& U & V &&&&& {} \\
	& W && X && W && X \\
	\\
	& Y && Z && Y && Z
	\arrow[from=2-2, to=2-4]
	\arrow[from=2-4, to=4-4]
	\arrow[from=2-2, to=4-2]
	\arrow[from=4-2, to=6-2]
	\arrow[from=6-2, to=6-4]
	\arrow[from=4-4, to=6-4]
	\arrow[from=4-2, to=4-4]
	\arrow[from=1-1, to=6-2]
	\arrow[from=1-1, to=2-4]
	\arrow[dotted, from=1-1, to=2-2]
	\arrow[curve={height=12pt}, from=2-5, to=6-6]
	\arrow[from=4-6, to=6-6]
	\arrow[from=6-6, to=6-8]
	\arrow[from=4-6, to=4-8]
	\arrow[from=4-8, to=6-8]
	\arrow[from=2-5, to=3-7]
	\arrow[from=3-7, to=4-8]
	\arrow[dotted, from=2-5, to=3-6]
	\arrow[from=3-6, to=4-6]
	\arrow[curve={height=-30pt}, from=2-5, to=4-8]
	\arrow[dotted, from=2-5, to=4-6]
	\arrow[from=3-6, to=3-7]
\end{tikzcd}\]
\caption{Example of the extra compositions we use in Exercise 3.}
\end{figure}

\vspace{0.5cm}
\textbf{Exercise} (1.3.S)
We will use this in the next one.

Assuming that maps $W \rightarrow X_1 \rightarrow Y$ and $W \rightarrow X_2 \rightarrow Y$ agree then maps $W \rightarrow X_1 \rightarrow Y \rightarrow Z$ $W \rightarrow X_2 \rightarrow Y \rightarrow Z$ must also agree. Thus by the unique property of $X_1 \times_Z X_2$ there exists a unique map from $X_1 \times_Y X_2 \rightarrow X_1 \times_Z X_2$.

\vspace{0.5cm}
\textbf{Exercise 4} (1.3.S)
Suppose we have the morphisms
$a: X_1 \rightarrow Y$,
$b: X_2 \rightarrow Y$ and
$c: Y \rightarrow Z$. 

There exists a natural morphism $x_{YZ}: X_1 \times_Y X_2 \rightarrow X_1 \times_Z X_2$ as proven in the previous exercise.

The morphism from $X_1 \times_Y X_2 \rightarrow Y$ exists by assumption of the fibered product. Call it $a \circ pr_{X1} = b \circ pr_{X2}$.

By assumption of fibered product $X_1 \times_Z X_2$ and morphisms $a$, $b$, there exists a map from $X_1 \times_Z X_2 \rightarrow Y$. By the universal property of $Y\times_Z Y$ there must be a unique map $\alpha: X_1 \times_Z X_2 \rightarrow Y \times_Z Y$.

Consider the set $YY =:\{(y,y)|y\in Y\}$. We have an obvious bijection between $YY$ and $Y$, and by the universal property of $Y \times_Z Y$ a unique map $j: YY \rightarrow Y \times_Z Y$. Thus a map $\beta = j \circ y \mapsto (y,y): Y \rightarrow Y \times_Z Y$ exists.

Suppose we have an object $W$ with maps to $Y$ and $X_1 \times_Z X_2$ s.t. the
maps agree in $Y \times_Z Y$. This is equal to giving maps from $W$ to $X_1$,
$X_2$ and $Y$ s.t. that the maps through $X_i$ argee in $Y$ (because we have a
morphism from $Y \times_Z Y \rightarrow Y$). 
By the universal property of $X_1 \times_Y X_2$ there exists unique map $W
\rightarrow X_1 \times_Y X_2$ and the Magic Diagram is cartesian.

Note: I may have made a mistake with calling some morphisms maps and vice versa.
This got me thinking wether we are assuming that all the maps are automatically
morphisms in these diagrams? Especially the maps from $W$ to $X_1$ and $X_2$ in
the definition of fibered product.
% https://q.uiver.app/?q=WzAsMjAsWzEsMSwiVSJdLFszLDEsIlYiXSxbMSwzLCJXIl0sWzMsMywiWCJdLFsxLDUsIlkiXSxbMyw1LCJaIl0sWzAsMCwiQSJdLFswLDIsIkIiXSxbNiwxLCJYXzEiXSxbNiwzLCJYXzIiXSxbOCwyLCJZIl0sWzEwLDIsIloiXSxbNiw1LCJYXzEgXFx0aW1lc19ZIFhfMiJdLFs4LDUsIlhfMSBcXHRpbWVzX1ogWF8yIl0sWzYsNywiWSJdLFs4LDcsIllcXHRpbWVzX1ogWSJdLFs4LDksIlkiXSxbNiw5LCJaIl0sWzUsNiwiWF8xIl0sWzUsNCwiVyJdLFswLDEsInByX3tWfSJdLFswLDIsInByX3tXfSJdLFsxLDMsIlxcYWxwaGEiXSxbMiwzLCJwcl97WH0sXFxiZXRhIl0sWzMsNSwiXFxhbHBoYSciXSxbMiw0LCJwcl97WX0iXSxbNCw1LCJcXGJldGEnIiwyXSxbNiwxLCJHIiwxLHsiY3VydmUiOi0xfV0sWzYsMiwicHJfe1d9XFxjaXJjIGYiLDEseyJjdXJ2ZSI6MX1dLFs2LDAsIlxcZXhpc3RzICEgZiIsMSx7InN0eWxlIjp7ImJvZHkiOnsibmFtZSI6ImRvdHRlZCJ9fX1dLFsxLDUsIlxcYWxwaGEnXFxjaXJjXFxhbHBoYSIsMCx7ImN1cnZlIjotM31dLFs3LDQsInByX3tZfVxcY2lyYyBnIiwxLHsiY3VydmUiOjF9XSxbNywzLCIiLDEseyJjdXJ2ZSI6LTF9XSxbNywyLCJcXGV4aXN0cyAhZyIsMSx7InN0eWxlIjp7ImJvZHkiOnsibmFtZSI6ImRvdHRlZCJ9fX1dLFs2LDQsIkY9cHJfe1l9XFxjaXJjIHByX3tXfVxcY2lyYyBmIiwyLHsiY3VydmUiOjV9XSxbNiwzLCIiLDEseyJjdXJ2ZSI6LTN9XSxbOCwxMCwiYSIsMV0sWzksMTAsImIiLDFdLFsxMCwxMSwiYyIsMV0sWzEyLDEzLCJpZCIsMV0sWzEyLDE0LCJhIFxcY2lyYyBwcl9YMSIsMV0sWzE0LDE1LCJwcl9ZIiwxLHsic3R5bGUiOnsidGFpbCI6eyJuYW1lIjoiYXJyb3doZWFkIn0sImhlYWQiOnsibmFtZSI6Im5vbmUifX19XSxbMTMsMTUsInlcXG1hcHN0byB5LHkgXFxjaXJjIGEgXFxjaXJjIHByX1gxIiwxXSxbMTQsMTcsImMiLDFdLFsxNiwxNywiYyIsMV0sWzE1LDE2LCJwcl9ZIiwxXSxbMTIsMThdLFsxOCwxNCwiYSIsMV0sWzEzLDE4XSxbMTQsMTUsInkgXFxtYXBzdG8geSx5IiwxLHsiY3VydmUiOjJ9XSxbMTksMTIsImYiLDEseyJzdHlsZSI6eyJib2R5Ijp7Im5hbWUiOiJkb3R0ZWQifX19XSxbMTksMTMsImYiLDEseyJjdXJ2ZSI6LTV9XV0=

\textbf{Exercise 5} (1.3.Y)

(a) In the case of C=A the composition $h_A \circ g$ gives is precisely the morphism
from $\mathrm{Mor}(A,A')$. I think I should show that the $i_A \circ h_A = h_A
\circ g$. Is this an obvious fact from the definition that the $g$ commutes with
the morphisms in $\mathrm{Mor}(C,A)$? 

(b) If $i_C$ are all bijections then we have inverse mappings $i_C^{-1}$ and a
morphism $i_C^{-1}: \mathrm{Mor}(A,A') \rightarrow {\mathrm{Mor}(A,A)}$. By
previous part we have the unique $g': A' \rightarrow A$.

This
means that $i_C^{-1} \circ i_C$ is the identity morphism $id_A:
\mathrm{Mor}(A,A) \rightarrow \mathrm{Mor}(A,A') \rightarrow \mathrm{Mor}(A,A)$.
By previous part we have that every $i_C$ is of the form $u \mapsto g \circ u$
and $i_C^{-1}$ of the form $u \mapsto g' \circ u$.
Thus $id_A = i^{-1}_A(i_A(id_A)) = i_A^{-1}(g\circ id_A)) = g'(g\circ id_A)) =
g' \circ g$.
Similarly by setting $A=A'$ we get that $id_{A'} = g \circ g'$. This implies
that $g$ is an isomorphism.
\end{document}
